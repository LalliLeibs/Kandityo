\DocumentMetadata{testphase={phase-III,math,tabular}, pdfversion=2.0, pdfstandard=A-4}

\documentclass[utf8,english,accessibility]{gradu3}

\usepackage{graphicx}
\usepackage{amsmath}
\usepackage{booktabs}

\usepackage[bookmarksopen,bookmarksnumbered,linktocpage]{hyperref}

\addbibresource{malliopas.bib}

\begin{document}

\title{Accessibility annex}
\studyline{All study lines}
\abstract{%
  This annex provides some information on how to produce accessible PDFs with \LaTeX.
}

\author{Frankie Robertson}
\contactinformation{\texttt{frankie.r.robertson@jyu.fi}}
\supervisor{Unsupervised work}

\maketitle

\mainmatter

\chapter{Introduction}

Newer versions of \LaTeX are gaining support for producing tagged PDFs, which
include certain accessibility features. This short guide is intended to help
you get started with features. For more general guidance, including why
students are encouraged to produce tagged PDFs of their these, see the
information provided by the Open Science
Centre
\footnote{Available at \url{https://openscience.jyu.fi/en/thesis-tutorial/bachelors-masters-thesis/publishing-your-thesis/thesis-accessibility}}.

\chapter{Enabling accessibility features}

In order to produce tagged PDFs, you need to use a recent version of \LaTeX,
with a recent version of the experimental \texttt{tagpdf} package. TeX Live
2024 or equivalent should be sufficient. The easiest way to obtain this is
using an official installer or Docker images povided by Island Of
TeX\footnote{\url{https://github.com/islandoftex/texlive}}.

You should then add this line as the first line of your document:

\begingroup\footnotesize
\begin{verbatim}
\DocumentMetadata{testphase={phase-III,math,tabular}, pdfversion=2.0, pdfstandard=A-4}
\end{verbatim}
\endgroup

You also need to add the \texttt{accessibility} option to the \texttt{documentclass} line. For example:

\begingroup\footnotesize
\begin{verbatim}
\documentclass[utf8,english,accessibility]{gradu3}
\end{verbatim}
\endgroup

\chapter{Providing alternative text for images}

Figure \ref{fig:opus-kissa} is an example of an image with alternative text. The alternative text is provided in the \texttt{alt} attribute of the \texttt{includegraphics} command. This is the text that will be read out by a screen reader.

Note that both a caption and an alternative. While the caption explains the purpose of the image in a general way and in relation to the document, the alternative text is intended to --- as far as possible --- provide a replacement for the image. The image is included with the following code:

\begingroup\footnotesize
\begin{verbatim}
\includegraphics[
  height=5cm,
  keepaspectratio,
  alt={
  An abstract illustration of a monochrome cat. The black head is composed of a
  quarter circular sector. Super-imposed are triangular eyes, nose, and
  whiskers with a rounded mouth shape. A short tail winds out of a sharp edge
  at the bottom.
  }
]{opus-kissa}
\end{verbatim}
\endgroup

\begin{figure}[ht]\centering
\includegraphics[
  height=5cm,
  keepaspectratio,
  alt={
  An abstract illustration of a monochrome cat. The black head is composed of a
  quarter circular sector. Super-imposed are triangular eyes, nose, and
  whiskers with a rounded mouth shape. A short tail winds out of a sharp edge
  at the bottom.
  }
]{opus-kissa}
  \caption{A test image taken from \textcite{kaijanaho03:_latex_ams_latex} included to give an example of an image with alternative text.}
  \label{fig:opus-kissa}
\end{figure}

\printbibliography

\end{document}
