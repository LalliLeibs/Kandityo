\documentclass[utf8,bachelor] {gradu3}

\usepackage[bookmarksopen,bookmarksnumbered,linktocpage]{hyperref}

\addbibresource{kandi.bib} % Lähdetietokannan tiedostonimi

\begin{document}

\title{Otsikko}
\translatedtitle{Header}
\studyline{Koulutusteknologia}
\avainsanat{Avainsanat}
\keywords{Keywords}
\tiivistelma{%
Tiivistelmä
}
\abstract{%
Abstract
}

\author{Lauri Ahonen}
\contactinformation{\texttt{lauri.t.ahonen@student.jyu.fi}}
% jos useita tekijöitä, anna useampi \author-komento
\supervisor{Ohjaamaton työ}
% jos useita ohjaajia, anna useampi \supervisor-komento

\maketitle

\mainmatter

\chapter{Johdanto}

Tutkielman varsinainen teksti alkaa aina luvulla ''Johdanto''.  Sen
kirjoittamisen voi hyvin jättää aivan tutkielman kirjoitusprosessin
loppuvaiheisiin.

\chapter{Tutkielman rakenne}

Yhteensä tutkielmassa on hyvä olla 5--9 numeroitua
lukua, siis Johdanto ja Yhteenveto mukaan lukien.  Tarvittaessa voit
käyttää alilukuja tarkempaan jäsentelyyn.\parencite{mustafa2023psychological}

\chapter{Yhteenveto}
Tutkielman viimeinen luku on Yhteenveto.  Sen on hyvä olla lyhyt;
siinä todetaan, mitä tutkielmassa esitetyn nojalla voidaan sanoa
johdannon väitteen totuudesta tai tutkimuskysymyksen vastauksesta
Yhteenvedossa tuodaan myös esille tutkielman heikkoudet (erityisesti
tekijät, jotka heikentävät tutkielman tulosten luotettavuutta), ellei
niitä ole jo aiemmin tuotu esiin esimerkiksi Pohdinta-luvussa.  Tässä
luvussa voidaan myös tuoda esille, mitä tutkimusta olisi tämän
tutkielman tulosten valossa syytä tehdä seuraavaksi.\parencite{ozccetin2019relationships}

\printbibliography
\end{document}
