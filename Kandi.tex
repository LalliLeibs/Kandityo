\documentclass[utf8,bachelor]{gradu3}
% Jos työ on kandidaatintutkielma eikä pro gradu, käytä ylläolevan asemesta
%\documentclass[utf8,bachelor]{gradu3}
% Jos kirjoitat englanniksi, käytä ylläolevan asemesta
%\documentclass[utf8,english]{gradu3}
% tai
%\documentclass[utf8,bachelor,english]{gradu3}

\usepackage{graphicx} % kuvien mukaan ottamista varten

\usepackage{amsmath} % hyödyllinen jos tekstisi sisältää matikkaa,
                     % ei pakollinen

\usepackage{booktabs} % hyvä kauniiden taulukoiden tekemiseen

% HUOM! Tämän tulee olla viimeinen \usepackage koko dokumentissa!
\usepackage[bookmarksopen,bookmarksnumbered,linktocpage]{hyperref}

\addbibresource{kandi.bib} % Lähdetietokannan tiedostonimi

\begin{document}

\title{Videopelien vaikutus peruskoululaisten kognitiivisiin taitoihin}
\author{Lauri Ahonen}
\contactinformation{\texttt{lauri.t.ahonen@student.jyu.fi}}
% jos useita tekijöitä, anna useampi \author-komento
%\supervisor{Ohjaamaton työ}
% jos useita ohjaajia, anna useampi \supervisor-komento

\subject{Kandidaatintutkielman}
%\type{tutkimussuunnitelma}

\maketitle

\mainmatter

\chapter{Johdanto}
Videopelaaminen on nykyajan nuorten keskuudessa suosittu vapaa-ajan harrastus. Vuoden 2022 pelaajabarometrin mukaan 10-19-vuotiaista nuorista 76\% pelaa digitaalisia pelejä viikoittain. \parencite{kinnunen2022pelaajabarometri} Kyseessä on siis laaja ilmiö, joka koskettaa suurinta osaa nuorista päivittäin. Laajan yleisön lisäksi videopelit ovat 2000-luvun aikana saavuttaneet ison roolin osana populaarikulttuuria ja ovatkin monen nuoren elämässä iso osa heidän identiteettiään. Teknologian kehityksen johdosta videopelaaminen on mahdollista lähes jokaiselle nuorelle, ja älypuhelimen omistajana pelit kulkevat aina käyttäjänsä mukana.  

Peruskouluaika on nuoren elämässä pitkä ja merkityksellinen elämänvaihe. ''Perusopetuksen tavoitteena on tukea oppilaiden kasvua ihmisinä ja yhteiskunnan jäseninä sekä opettaa tarpeellisia tietoja ja taitoja''.\parencite{OPH2025} Vaikka nykyaikana koulutusjärjestelmä onkin pyrkinyt muuttumaan akateemisesta oppimisesta laajemman ja moniulotteisemman osaamisen opetukseen, on perinteisten taitojen kuten lukemisen, laskemisen ja oppimisen taitojen opiskelu yhä koulutuksen keskiössä. Tämän oppimisen keskiössä on kognitiiviset taidot, jotka mahdollistavat tiedon syvällisen omaksumisen ja kriittisen ajattelun kehittymisen. 

Peruskouluikäiset lapset ovat kognitiivisen kehityksen kannalta kriittisessä elämänvaiheessa. Kognitio on moniulotteinen kokonaisuus, mutta tässä työssä käytän kognitiiviselle toiminnalle seuraavaa määritelmää: “Laajasti ymmärrettynä tiedonkäsittelyyn liittyvät toiminnot kuten havaitseminen, oppiminen, tarkkaivaisuus, muisti, päättely, sosiaalinen kognitio, kieli ja tietoisuus”. \parencite{hamalainen2006mieli} Nämä prosessit eivät kehity tyhjiössä, vaan vuorovaikutuksessa sisäisten sekä ulkoisten vaikutteiden johdosta. 

Tämän työn tarkoituksena on perehtyä siihen, kuinka videopelaaminen vaikuttaa peruskouluikäisten nuorten kognitiivisiin taitoihin. Tarkoituksena on kirjallisuuskatsauksen keinoin perehtyä aiheesta tehtyihin tutkimuksiin sekä tutkimuskirjallisuuteen ja muodostaa yleismaailmallinen käsitys aineistojen löydöksistä. Tässä työssä keskitytään oppimisen kannalta tärkeisiin kognition osa-alueisiin. Tutkimuskysymykset ovat seuraavat: 
\newpage
\begin{enumerate}
\item Kuinka säännöllinen videopelaaminen vaikuttaa peruskouluikäisten kognitiivisiin taitoihin? 
\item Miten videopelaamisen määrä tai pelin sisältö vaikuttaa kognitiivisiin taitoihin? 
\end{enumerate}

Lähdemateriaalina toimivat pääasiassa tutkimukset videopelien vaikutuksista kognition eri osa-alueisiin. Materiaalista on rajattu pois tutkimukset, joiden osallistujien ikä ei vastaa suomalaisen peruskoulujärjestelmän ikävuosia 7–16. Tämän lisäksi tutkimuksen päätarkoituksena on pitänyt olla videopelaamisen vaikutusten tarkastelu tiettyyn tai useampaan kognition osa-alueeseen. Kognition teoriakehyksen osalta lähteenä toimivat kehityspsykologiaa käsittelevät tieteelliset artikkelit sekä painetut kirjalliset lähteet.  Keskeisimpänä lähteiden etsinnän tietokantana on toiminut Google Scholar hakukone.  

\chapter{Videopelit ja peruskoululaiset}
Videopelaaminen on 2000-luvun aikana saavuttanut valtavan suosion ajanvietteenä sekä osana populaarikulttuuria erityisesti nuorten keskuudessa. Vuonna 1991 noin 45 \% pojista ilmoitti pelaavansa digitaalisia pelejä vähintään kerran vuodessa, kun taas 2017 luku on jo yli 95 \%. Tyttöjen osalta luku on kasvanut samalla aikavälillä yli 56 \%  \parencite{tilk2017}. Digitaalisten pelien saavutettavuus on kasvanut huomattavasti teknologisen kehityksen myötä ja nykyaikaiset mobiililaitteet mahdollistavat pelaamisen missä tahansa. Internetin leviäminen on mahdollistanut yksinpelaamisen laajentumisen osaksi sosiaalista verkostoa, jossa pelaajat voivat olla vuorovaikutuksessa toisiinsa ympäri maailman. Pelaamisen lisäksi pelikulttuuriin liittyy vahvasti erilaiset pelitapahtumat, messut, pelivideoiden ja striimien kuluttaminen sekä e-urheilu. Näiden tekijöiden myötä videopelaamisesta on tullut vakituinen osa nykyaikaista nuorisokulttuuria, jonka vaikutukset näkyvät nuorten jokaisella elämän osa-alueella.  

Tässä luvussa käsitellään lyhyesti videopelien kehityskaarta. Tämän jälkeen selvitetään, millainen on nykyaikainen pelikulttuuri peruskouluikäisten keskuudessa.

\section{Videopelien historiasta}

Videopelien kehityskaari on pitkä ja täynnä teknologisia läpimurtoja sekä kulttuurillisia vallankumouksia. Ensimmäiset videopelit kuten \textit{Tennis for Two} (1958) ja  \textit{Spacewar!} (1962) olivat tieteellisiä kokeiluita, jotka herättivät mielenkiinnon tietokoneen interaktiivisen käytön mahdollisuuksista. 1970-luvulla kotikonsolit ja pelihallit (Arcade) alkoivat yleistyä ja toivat pelaamisen osaksi sen aikaista viihdekulttuuria. 1980-lukua pidetään kolikkopelien kultaisena aikakautena. 80-luvun pelit kuten Donkey Kong ja Pac Man ovat tunnistettavia tavaramerkkejä yhä nykypäivänä. \parencite {stanton2015brief}

1980-luvun loppupuolella konsolit kuten Nintendo Entertainment System (NES) ja Sega Master System toivat videopelaamisen laajan yleisön koteihin. Edistykset kuten 3D-grafiikka ja moninpeli vauhdittivat suosion laajenemista entisestään. 2000-luvun pelikonsolit kuten Playstation 2 ja Xbox olivat maailmanlaajuisia hittejä. Myös kotitietokonoilla pelaaminen nousi entistä suurempaan suosioon.  Vuonna 2004 julkaistu Wold of Warcraft keräsi parhaimmillaan 12 miljoonaa samanaikaista tilaajaa vuonna 2010 \parencite{Wowlost}. \parencite {stanton2015brief}

2010-luvulla mobiilipelaaminen ja striimauspalvelut mullistivat videopelialan, ja e-urheilusta tuli ammattimainen kilpailumuoto, jossa pelit kuten League of Legends ja Counter-Strike nousivat maailmanlaajuiseen suosioon. Pelkästään League of Legends pelin maailmanmestaruuden finaalia seurasi parhaimmillaan 6.8 miljoonaa katsojaa \parencite{esportchart}.  

\section{Mitä, miksi ja miten nuoret pelaavat?}

Videopelit ovat nykypäivänä osa kasvavan lapsen kulttuuriympäristöä ja nuori törmää välttämättömästi pelaamiseen ilmiönä jo nuoruutensa ensimetreillä. Lähes 80 \% 10-19 vuotiaista nuorista kertoo pelaavansa videopelejä viikoittain \parencite{kinnunen2022pelaajabarometri}. Kuitenkin vuoden 2022 peruskouluissa teetetyn koululaiskyselyn mukaan, nuoret eivät ilmoittaneet videopelaamista toiveharrastuksekseen tai kertoneet halusta kokeilla videopelejä \parencite{koululaiskysely2022}. Videopelaamista ei siis ainakaan nuorten keskuudessa mielletä harrastuksena, vaan se rinnastetaan muun viihteen kulutuksen tavoin vain tavaksi viettää vapaa-aikaa.  

Mutta millaisista peleistä nykyajan nuoret ovat kiinnostuneet? Mannerheimin Lastensuojeluliiton Nuortennetin käyttäjät listasivat omaan TOP 10 listaansa seuraavat pelit: Minecraft, The Sims 4, Among Us, Momio, Five Nights at Freddy’s, CS:GO, Fortnite, Taiko No Tatsujin, Pokemon ja Fifa \parencite{nuortennettiparas}. Nuoret ovat siis kiinnostuneet laajasti eri genren peleistä.  

Pelialustojen näkökulmasta nuorten tottumukset eroavat muusta väestöstä huomattavasti. Käsikonsolit, kotikonsolit ja tietokoneet ovat niin nuorten, kuin muun väestön suosiossa tasaisesti, mutta mobiililaitteiden käyttö pelialustana on 10–19-vuotiaiden keskuudessa huomattavasti muuta väestöä yleisempää. Myös selainpelien suosio on nuorten keskuudessa yleistä. Nuorten videopelaaminen ei ole siis rajoittunut paikkaan, vaan pelaamista voidaan harrastaa kouluissa, muissa harrastuksissa tai kavereiden luona. \parencite{koululaiskysely2022} 

Vuosina 2019–2021 11–16-vuotiailta kerätyn aineiston perusteella, nuorille tärkeitä motivaation lähteitä pelaamiselle olivat mahdollisuus kehittyä peleissä, rentoutuminen ja oman ajan saaminen. Myös oman pahan olon helpottaminen ja pelien käyttäminen tunteiden hallintakeinona oli hyvin yleistä. Myös kilpailullisuus, haasteet ja onnistumisen kokemukset nousivat aineistossa esille huomattavina motivaation lähteinä. \parencite{laakso2023lasten}  

Pelaaminen on nuorille tärkeää sosiaalisena toimintana. Nuoret pelaavat huomattavasti enemmän tuttujen, kuin tuntemattomien kanssa. Yhdessä pelaaminen on yleistä kouluissa, harrastuksissa ja vapaa-ajalla. Nuoret hyödyntävät myös etäviestintä palveluita pelatakseen yhdessä, vaikka eivät olisikaan läsnä toistensa luona. Tästä kertoo myös korkea kyllä vastausten määrä tutkimuksen kysymykseen “Pelaan pitääkseni yhteyttä ystäviini”. Pelaaminen on siis nuorille keino sosialisoitua ja ylläpitää jo muodostuneita suhteita. \parencite{laakso2023lasten} 

Nuorten videopelikulttuuri on hyvin sukupuolittunutta. Alakouluikäisten keskuudessa pelaaminen on tasaisen yleistä kaikkien sukupuolten keskuudessa. Yläkouluasteelle siirtyessä ja sen aikana, aktiivisesti pelaavien tyttöjen määrä vähenee merkittävästi. Muiden sukupuolten osalta merkittävää muutosta ei tapahdu. Tämän lisäksi tytöt pelaavat poikia enemmän yksin, kun taas pojat pelaavat useammin tuttujen ja tuntemattomien pelaajien kanssa. Laakson tutkimuksesta voidaan myös huomata, että pojat pelaavat muita sukupuolia huomattavasti enemmän.  Poikien ja ei-binääristen vastaajien keskiarvo on 7-14 tuntia viikossa, kun taas tytöillä keskiarvo on 2-7 tuntia. \parencite{laakso2023lasten} 

Nuorten keskuudessa pelien kulttuurinen merkitys näkyy myös pelaamisen ulkopuolella. Monille nuorille pelaaminen on iso osa omaa identiteettiä ja tätä halutaan viestiä aktiivisesti muille. Nuoret osallistuvatkin aktiivisesti laajempaan pelikulttuuriin kuten striimien katsomiseen ja tuottamiseen, pelien muokkaukseen ja omien pelien luomiseen. Nuoret siis haluavat vaikuttaa omaan ympäristöönsä aktiivisesti ja kokevat kuuluvansa johonkin isompaan videopelien kuluttajina. maailmalle.\parencite{laakso2023lasten}  

\chapter{Kognitiiviset taidot}
- Mitä ne ovat
- Miten kehittyvät
-Miksi erityisen tärkeitä koululaisille

\chapter{Videopelaamisen vaikutukset kognitiivisiin taitoihin}

Kuten tässä työssä on aikaisemmin todettu, videopelit ovat peruskouluikäisten keskuudessa suosittu ajanviete ja iso osa nuorisokulttuuria. Nuoren pääasiallinen rooli yhteiskunnassa on kuitenkin koulunkäynti. Koulutuksen kannalta onkin tärkeää tutkia nuorten elämän muita osa-alueita ja niiden vaikutuksia koulunkäyntiin. Tutkijat \textcite{kovess2016time} huomasivat artikkelissa “Is time spent playing video games associated with mental health, cognitive and social skills in young children? “, että yli 5 tuntia viikossa pelaavat nuoret pärjäsivät keskimääräistä paremmin koulussa ja osoittivat korkeampaa älyllistä toimintakykyä. Toisaalta artikkeli “ Video gaming in school children: How much is enough? “ \parencite{pujol2016video} huomasi pelaamisen aiheuttavan keskittymishaasteita. Tutkimukset tuovatkin esiin tärkeän laajemman aiheen jota tässä luvussa on tarkoituksena käsitellä. Miten ja mihin peruskoululaisten kognitiivisiin taitoihin videopelit vaikuttavat? 

\section{Työmuisti}

Videopelaamisen positiivisia vaikutuksia työmuistiin on havaittu useissa tutkimuksissa. Tutkimuksessa “Association of Video Gaming With Cognitive Performance Among Children “ \parencite{jamanetworkopen} käytettiin fMRI-kuvantamista ja kognitiivisia testejä videopelaajien ja pelaamattomien 9-10-vuotiaiden lasten aivotoiminnan tutkimiseen. Testeissä huomattiin, että videopelaajat suoriuituivat 2-Back työmuistitehtävässä merkittävästi paremmin. Tämän lisäksi aivokuvantaminen havaitsi aktiivisempaa toimintaa työmuistiin liittyvillä aivoalueilla. Havaintoa tukee vuoden 2020 tutkimus, jossa videopelaajat suoriuituivat työmuistia testaavassa Digit Span -testissä merkittävästi paremmin \parencite{choudhury2022cognitive}. Molemmissa tutkimuksissa erot olivat tilastollisesti merkittäviä. Näihin havaintoihin perustuen voidaankin todeta, että videopelit voivat kehittää työmuistin toimintaa. 

Mitkä videopelien mekanismit saattavat mahdollistaa edellä mainitun vaikutuksen? Videopelit vaativat nopeaa tiedonkäsittelyä ja useiden tehtävien samanaikaista hallintaa. Tämä tarjoaa pelaajalleen kognitiivisia haasteita jotka vuorostaan harjaannuttavat käyttäjänsä työmuistia \parencite{choudhury2022cognitive}. Tämän lisäksi videopelit vaativat myös jatkuvaa tarkkaavaisuuden vaihtamista ja visuaalisten ärsykkeiden käsittelyä, joka voi harjaannuttaa pelaajan työmuistin kapasiteettia \parencite{jamanetworkopen}.  

Työmuistin kannalta pelin genrellä on vaikutusta sen tuomiin hyötyihin. Videopelaaminen yleisesti parantaa työmuistin toimivuutta, mutta pulma- ja strategiapelit kehittävät tätä kognition osa-aluetta tehokkaammin kuin muut peligenret.  Tämän lisäksi pelaajan yleinen taitotaso videopeleissä osoittaa tilastollista korrelaatiota työmuistin kehityksen kanssa. Mitä taitavampi pelaaja on, sitä paremmin hän suoruitui tutkimuksen muistitehtävissä. \parencite {zioga2024validation} 

Useat tutkimukset osoittavat että videopelit vaikuttavat postitiivisesti erityisesti lyhytkestoiseen työmuistiin. Löydöksiin on kuitenkin suhtauduttava varauksellisesti. Erityisesti työmuistin kehityksen siirtymisestä pitkäaikaisesti varsinaisiin oppimistilanteisiin ei ole tutkimusta. Tämän lisäksi peligenrellä ja pelaamiseen käytettävällä ajalla on suuri vaikutus saavutettaviin hyötyihin.   

\section{Visuaalinen hahmotus - ja käsittelykyky}

Visuaalinen hahmotus- ja käsittelykyky ovat tärkeä osa kognitiivisia prosesseja. Videopelien vaikutuksista näihin kykyihin on tehty useita tutkimuksia, jotka viittaavat videopelien parantavan näitä taitoja. Tutkijat \textcite{jamanetworkopen} havaitsivat fMRI analyysissa visuaalisen prosessoinnin kannalta keskeisillä aivoalueilla signaalieroja pelaajien ja ei-pelaajien välillä. Tämä viittaa siihen, että videopelaaminen voi muokata hermoyhteyksiä, joiden tehtävä on visuaalinen prosessointi. Lisäksi tutkimuksissa on havaittu, että erityisesti FPS ja pulmanratkaisupelejä pelaavat nuoret pärjäsivät visuaalista hahmotuskykyä mittaavissa kokeissa tavallista paremmin \parencite{zioga2024validation}.  

Vaikutuksia arvioidessa tulee ottaa huomioon useita eri tekijöitä pelkän pelaamisen tai pelaamattomuuden lisäksi. Esimerkiksi vaikka vähintään tunnin viikossa pelaavien psykomotooriset vasteet olivat nopeampia ja tarkempia, eivät tulokset parantuneet pelaamisen määrän kasvaessa \parencite{pujol2016video}. Tutkimuksessa havaitut haitat kuitenkin kasvoivat pelaamisen määrän mukana, joten nuorten peliajan hallinta on tärkeässä roliissa hyötyjen maksimoinnissa. 

Vaikka tämän osa-alueen löydökset ovat lupaavia, videopelit eivät näytä aiheuttavan suuria höytyjä visuaaliselle hahmotus- ja käsittelykyvylle. Tutkimusten havaitsemat parannukset kognition osa-alueella olivat huomattavissa, mutta erot pelaajien ja ei-pelaajien välillä eivät olleet isoja. Höydyt on kuitenkin havaittu useissa tutkimuksissa, joten videopelit osaltaan voivat parantaa tätä kognition osa-aluetta. Videopelit ovat usein nopeatempoisia ja vaativat dynaamista tarkkaavaisuuden hallintaa. Nämä mekanismit voivat toimia osaltaan tätä kognition osa-aluetta kehittävänä tekijänä. Tutkimusta tarvitaan kuitenkin enemmän ja eri genrejen toimimista parempina harjoitusalustoina tulisi tutkia tarkemmin.  

\chapter{Yhteenveto}

\printbibliography

\end{document}
