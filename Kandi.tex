\documentclass[utf8,bachelor]{gradu3}
% Jos työ on kandidaatintutkielma eikä pro gradu, käytä ylläolevan asemesta
%\documentclass[utf8,bachelor]{gradu3}
% Jos kirjoitat englanniksi, käytä ylläolevan asemesta
%\documentclass[utf8,english]{gradu3}
% tai
%\documentclass[utf8,bachelor,english]{gradu3}

\usepackage{graphicx} % kuvien mukaan ottamista varten

\usepackage{amsmath} % hyödyllinen jos tekstisi sisältää matikkaa,
                     % ei pakollinen

\usepackage{booktabs} % hyvä kauniiden taulukoiden tekemiseen

% HUOM! Tämän tulee olla viimeinen \usepackage koko dokumentissa!
\usepackage[bookmarksopen,bookmarksnumbered,linktocpage]{hyperref}

\addbibresource{kandi.bib} % Lähdetietokannan tiedostonimi

\begin{document}

\title{Videopelien vaikutus peruskoululaisten kognitiivisiin taitoihin}
\author{Lauri Ahonen}
\contactinformation{\texttt{lauri.t.ahonen@student.jyu.fi}}
% jos useita tekijöitä, anna useampi \author-komento
%\supervisor{Ohjaamaton työ}
% jos useita ohjaajia, anna useampi \supervisor-komento

\subject{Kandidaatintutkielman}
%\type{tutkimussuunnitelma}

\maketitle

\mainmatter

\chapter{Johdanto}
Videopelaaminen on nykyajan nuorten keskuudessa suosittu vapaa-ajan harrastus. Vuoden 2022 pelaajabarometrin mukaan 10-19-vuotiaista nuorista 76\% pelaa digitaalisia pelejä viikoittain. \parencite{kinnunen2022pelaajabarometri} Kyseessä on siis laaja ilmiö, joka koskettaa suurinta osaa nuorista päivittäin. Laajan yleisön lisäksi videopelit ovat 2000-luvun aikana saavuttaneet ison roolin osana populaarikulttuuria ja ovatkin monen nuoren elämässä iso osa heidän identiteettiään. Teknologian kehityksen johdosta videopelaaminen on mahdollista lähes jokaiselle nuorelle, ja älypuhelimen omistajana pelit kulkevat aina käyttäjänsä mukana.  

Peruskouluaika on nuoren elämässä pitkä ja merkityksellinen elämänvaihe. ''Perusopetuksen tavoitteena on tukea oppilaiden kasvua ihmisinä ja yhteiskunnan jäseninä sekä opettaa tarpeellisia tietoja ja taitoja''.\parencite{OPH2025} Vaikka nykyaikana koulutusjärjestelmä onkin pyrkinyt muuttumaan akateemisesta oppimisesta laajemman ja moniulotteisemman osaamisen opetukseen, on perinteisten taitojen kuten lukemisen, laskemisen ja oppimisen taitojen opiskelu yhä koulutuksen keskiössä. Tämän oppimisen keskiössä on kognitiiviset taidot, jotka mahdollistavat tiedon syvällisen omaksumisen ja kriittisen ajattelun kehittymisen. 

Peruskouluikäiset lapset ovat kognitiivisen kehityksen kannalta kriittisessä elämänvaiheessa. Kognitio on moniulotteinen kokonaisuus, mutta tässä työssä käytän kognitiiviselle toiminnalle seuraavaa määritelmää: “Laajasti ymmärrettynä tiedonkäsittelyyn liittyvät toiminnot kuten havaitseminen, oppiminen, tarkkaivaisuus, muisti, päättely, sosiaalinen kognitio, kieli ja tietoisuus”. \parencite{hamalainen2006mieli} Nämä prosessit eivät kehity tyhjiössä, vaan vuorovaikutuksessa sisäisten sekä ulkoisten vaikutteiden johdosta. 

Tämän työn tarkoituksena on perehtyä siihen, kuinka videopelaaminen vaikuttaa peruskouluikäisten nuorten kognitiivisiin taitoihin. Tarkoituksena on kirjallisuuskatsauksen keinoin perehtyä aiheesta tehtyihin tutkimuksiin sekä tutkimuskirjallisuuteen ja muodostaa yleismaailmallinen käsitys aineistojen löydöksistä. Tässä työssä keskitytään oppimisen kannalta tärkeisiin kognition osa-alueisiin. Tutkimuskysymykset ovat seuraavat: 
\newpage
\begin{enumerate}
\item Kuinka säännöllinen videopelaaminen vaikuttaa peruskouluikäisten kognitiivisiin taitoihin? 
\item Miten videopelaamisen määrä tai pelin sisältö vaikuttaa kognitiivisiin taitoihin? 
\end{enumerate}

Lähdemateriaalina toimivat pääasiassa tutkimukset videopelien vaikutuksista kognition eri osa-alueisiin. Materiaalista on rajattu pois tutkimukset, joiden osallistujien ikä ei vastaa suomalaisen peruskoulujärjestelmän ikävuosia 7–16. Tämän lisäksi tutkimuksen päätarkoituksena on pitänyt olla videopelaamisen vaikutusten tarkastelu tiettyyn tai useampaan kognition osa-alueeseen. Kognition teoriakehyksen osalta lähteenä toimivat kehityspsykologiaa käsittelevät tieteelliset artikkelit sekä painetut kirjalliset lähteet.  Keskeisimpänä lähteiden etsinnän tietokantana on toiminut Google Scholar hakukone.  

\chapter{Videopelit ja peruskoululaiset}
Videopelaaminen on 2000-luvun aikana saavuttanut valtavan suosion ajanvietteenä sekä osana populaarikulttuuria erityisesti nuorten keskuudessa. Vuonna 1991 noin 45 \% pojista ilmoitti pelaavansa digitaalisia pelejä vähintään kerran vuodessa, kun taas 2017 luku on jo yli 95 \%. Tyttöjen osalta luku on kasvanut samalla aikavälillä yli 56 \%  \parencite{tilk2017}. Digitaalisten pelien saavutettavuus on kasvanut huomattavasti teknologisen kehityksen myötä ja nykyaikaiset mobiililaitteet mahdollistavat pelaamisen missä tahansa. Internetin leviäminen on mahdollistanut yksinpelaamisen laajentumisen osaksi sosiaalista verkostoa, jossa pelaajat voivat olla vuorovaikutuksessa toisiinsa ympäri maailman. Pelaamisen lisäksi pelikulttuuriin liittyy vahvasti erilaiset pelitapahtumat, messut, pelivideoiden ja striimien kuluttaminen sekä e-urheilu. Näiden tekijöiden myötä videopelaamisesta on tullut vakituinen osa nykyaikaista nuorisokulttuuria, jonka vaikutukset näkyvät nuorten jokaisella elämän osa-alueella.  

Tässä luvussa käsitellään lyhyesti videopelien kehityskaarta. Tämän jälkeen selvitetään, millainen on nykyaikainen pelikulttuuri peruskouluikäisten keskuudessa.

\section{Videopelien historiasta}

Videopelien kehityskaari on pitkä ja täynnä teknologisia läpimurtoja sekä kulttuurillisia vallankumouksia. Ensimmäiset videopelit kuten \textit{Tennis for Two} (1958) ja  \textit{Spacewar!} (1962) olivat tieteellisiä kokeiluita, jotka herättivät mielenkiinnon tietokoneen interaktiivisen käytön mahdollisuuksista. 1970-luvulla kotikonsolit ja pelihallit (Arcade) alkoivat yleistyä ja toivat pelaamisen osaksi sen aikaista viihdekulttuuria. 1980-lukua pidetään kolikkopelien kultaisena aikakautena. 80-luvun pelit kuten Donkey Kong ja Pac Man ovat tunnistettavia tavaramerkkejä yhä nykypäivänä. \parencite {stanton2015brief}

1980-luvun loppupuolella konsolit kuten Nintendo Entertainment System (NES) ja Sega Master System toivat videopelaamisen laajan yleisön koteihin. Edistykset kuten 3D-grafiikka ja moninpeli vauhdittivat suosion laajenemista entisestään. 2000-luvun pelikonsolit kuten Playstation 2 ja Xbox olivat maailmanlaajuisia hittejä. Myös kotitietokonoilla pelaaminen nousi entistä suurempaan suosioon.  Vuonna 2004 julkaistu Wold of Warcraft keräsi parhaimmillaan 12 miljoonaa samanaikaista tilaajaa vuonna 2010 \parencite{Wowlost}. \parencite {stanton2015brief}

2010-luvulla mobiilipelaaminen ja striimauspalvelut mullistivat videopelialan, ja e-urheilusta tuli ammattimainen kilpailumuoto, jossa pelit kuten League of Legends ja Counter-Strike nousivat maailmanlaajuiseen suosioon. Pelkästään League of Legends pelin maailmanmestaruuden finaalia seurasi parhaimmillaan 6.8 miljoonaa katsojaa \parencite{esportchart}.  

\section{Mitä, miksi ja miten nuoret pelaavat?}

Videopelit ovat nykypäivänä osa kasvavan lapsen kulttuuriympäristöä ja nuori törmää välttämättömästi pelaamiseen ilmiönä jo nuoruutensa ensimetreillä. Tutustuminen peleihin voi tapahtua oman perheen ja kavereiden kautta tai internetissä. Mutta millaisista peleistä nykyajan nuoret ovat kiinnostuneet? Mannerheimin Lastensuojeluliiton Nuortennetin käyttäjät listasivat omaan TOP 10 listaansa seuraavat pelit: Minecraft, The Sims 4, Among Us, Momio, Five Nights at Freddy’s, CS:GO, Fortnite, Taiko No Tatsujin, Pokemon ja Fifa \parencite{nuortennettiparas}. Nuoret ovat siis kiinnostuneet laajasti eri genren peleistä aina sotapeleistä musiikkipeleihin.  

Vuosina 2019–2021 11–16-vuotiailta kerätyn aineiston perusteella, nuorille tärkeitä motivaation lähteitä pelaamiselle olivat mahdollisuus kehittyä, rentoutuminen ja oman ajan saaminen. Myös oman pahan olon helpottaminen ja pelien käyttäminen tunteiden hallintakeinona oli hyvin yleistä. Myös kilpailullisuus, haasteet ja onnistumisen kokemukset nousivat aineistossa esille huomattavina motivaation lähteinä. \parencite{laakso2023lasten} 

Pelaaminen on nuorille tärkeää sosiaalisena toimintana. Nuoret pelaavat huomattavasti enemmän tuttujen, kuin tuntemattomien kanssa. Yhdessä pelaaminen on yleistä koulussa, harrastuksissa ja vapaa-ajalla. Nuoret hyödyntävät myös etäviestintä palveluita pelatakseen yhdessä, vaikka eivät olisikaan läsnä toistensa luona. Tästä kertoo myös korkea samanmielisyys tutkimuksen kysymykseen “Pelaan pitääkseni yhteyttä ystäviini”. Pelaaminen on siis nuorille keino sosialisoitua ja ylläpitää jo muodostuneita suhteita. \parencite{laakso2023lasten} 

\chapter{Kognitiiviset taidot}
- Mitä ne ovat
- Miten kehittyvät
-Miksi erityisen tärkeitä koululaisille

\chapter{Videopelaamisen vaikutukset kognitiivisiin taitoihin}

\section{Hyödyt}
- Mitä mahdollisia hyötyjä
-Voiko pelaamista siis optiomoida näiden saamiseksi

\section{Haitat}
- Mitä haittoja
-Miten vältetään tai vähennetään

\chapter{Yhteenveto}

\printbibliography

\end{document}
