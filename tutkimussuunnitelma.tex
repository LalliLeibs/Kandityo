\documentclass[utf8,bachelor]{gradu3}
% Jos työ on kandidaatintutkielma eikä pro gradu, käytä ylläolevan asemesta
%\documentclass[utf8,bachelor]{gradu3}
% Jos kirjoitat englanniksi, käytä ylläolevan asemesta
%\documentclass[utf8,english]{gradu3}
% tai
%\documentclass[utf8,bachelor,english]{gradu3}

\usepackage{graphicx} % kuvien mukaan ottamista varten

\usepackage{amsmath} % hyödyllinen jos tekstisi sisältää matikkaa,
                     % ei pakollinen

\usepackage{booktabs} % hyvä kauniiden taulukoiden tekemiseen

% HUOM! Tämän tulee olla viimeinen \usepackage koko dokumentissa!
\usepackage[bookmarksopen,bookmarksnumbered,linktocpage]{hyperref}

\addbibresource{tutkimussuunnitelma.bib} % Lähdetietokannan tiedostonimi

\begin{document}

\title{Tutkimussuunnitelma}
\author{Lauri Ahonen}
\contactinformation{\texttt{lauri.t.ahonen@student.jyu.fi}}
% jos useita tekijöitä, anna useampi \author-komento
%\supervisor{Ohjaamaton työ}
% jos useita ohjaajia, anna useampi \supervisor-komento

\subject{Kandidaatintutkielman}
\type{tutkimussuunnitelma}

\maketitle

\mainmatter

\chapter{Johdanto}

Videopelaaminen on yhä useamman nuoren vapaa-ajan harrastus. Tämän kandidaatin tutkielman tarkoituksena on tarkastella videopelaamisen vaikutuksia peruskouluikäisten nuorten kognitiivisiin taitoihin. Tässä tutkimuksessa perehdytään kirjallisuuskatsauksen avulla aihealueen tutkimuskirjallisuuteen aina 1980-luvulta nykypäivään. Työssä perehdytään aiempiin tutkimuksiin ja erotellaan pelaamisen vaikutuksia yksittäisiin kognition osa-alueisiin. Tämän avulla tarkoituksena on luoda lukijalle kokonaisvaltainen kuva videopelaamisen vaikutuksista lapsen oppimisen taitoihin.  

Suunnitelman luvussa kaksi esitellään aiheen keskeisimmät käsitteet ja tutustutaan katsauksen merkittävimpiin tutkimuksiin. Kolmannessa luvussa esitellään tutkimusaihe ja kysymys. Neljännen luvun tarkoitus on esitellä käytetty tutkimusmenetelmä. Viimeisessä luvussa perehdytään tutkimuksen merkityksellisyyteen.  

\chapter{Keskeiset käsitteet ja kirjallisuuskatsaus}

\section{Keskeiset käsitteet}

Kognitio on laajasti ymmärrettynä tiedonkäsittelyyn liittyvät toiminnot kuten havaitseminen, oppiminen, tarkkaivaisuus, muisti, päättely, sosiaalinen kognitio, kieli ja tietoisuus \parencite{hamalainen2006mieli}. 

Videopeli on elektroninen peli, jota pelataan digitaalisella päätelaitteella. Laite voi olla puhelin, tabletti, tietokone tai pelikonsoli.  

Pelillistäminen on prosessi, jossa pelien ominaisuuksia hyödynnetään opetustarkoituksessa. Tarkoituksena on yleensä hyödyntää pelien elämyksellisyyttä luomaan mielekäs kokemus oppimisprosessille.

\section{Kirjallisuuskatsaus}

Videopelien vaikutuksia käyttäjiin ja erityisesti nuoriin on alettu tutkimaan systemaattisesti 1980-luvulta alkaen. Jo 1980-luvun aikana kotikonsolit kuten Nintendo Entertainment System ja Atari toivat videopelit laajan yleisön saataville. Yksi varhaisimmista tutkimuksista ''Videogames, television violence, and aggression in teenagers'' tarkastelee videopelaamisen vaikutuksia lasten aggressiivisuuteen \parencite{dominick1984videogames}. Varhaisen tutkimuksen tulokset olivat ristiriitaisia, mutta kiinnostus aihetta kohtaan oli herätelty.  

Videopelien suosion kasvaessa, on myös niiden vaikutusten tutkimista jatkettu enenevissä määrin. Videopelien vaikutuksia kognitiivisiin taitoihin tarkasteltiin ensimmäisiä kertoja artikkeleissa ''Videogames as cultural artifacts'' \parencite{greenfield1994video} ja ''Action video game modifies visual selective attention'' \parencite{green2003action}. Tutkimukset toivat esiin videopelien positiivisia vaikutuksia esimerkiksi hahmotuskykyyn ja ongelmanratkaisutaitoihin.  

Videopelit vetoavat nuoriin. Vuoden 2022 pelaajabarometrin mukaan 10–19-vuotiaista nuorista 76 \% pelaa digitaalisia pelejä viikoittain \parencite{kinnunen2022pelaajabarometri}. Videopelien vaikutuksia tähän ikäryhmään on tarkasteltu lähivuosina kasvavalla määrällä. Tutkimukset: ''Is time spent playing video games associated with mental health, cognitive and social skills in young children?'' \parencite{kovess2016time} ja ''The relationships between video game experience and cognitive abilities in adolescents'' \parencite{ozccetin2019relationships} tarkastelevat videopelien vaikutuksia nuoriin kognitiivisten kykyjen näkökulmasta. Molemmat tutkimukset löysivät merkittäviä havaintoja videopelien vaikutuksista kognitiivisiin kykyihin.  

Kognitiivisen näkökulman lisäksi, myös suoraa oppimista videopeleistä on tutkittu. Tutkimuksessa ''Relevance of videogames in the learning and development of young children'' \parencite{zhao2015relevance} havaittiin videopeleistä oppimisen lisäksi myös sosiaalisen kehityksen mahdollisuuksia.  

\chapter{Tutkimusaihe ja tutkimuskysymys}

\section{Tutkimusaihe}

Tutkimukseni tarkoituksena selvittää kuinka videopelit vaikuttavat peruskouluikäisten lasten kognitiivisiin taitoihin.  Tarkastelen videopelien mahdollisia positiivisia, sekä negatiivisia  vaikutuksia muunmuassa muistiin, havainnointiin, tarkkaavaisuuteen, päättelyyn ja sosiaaliseen kognitioon.  Kohderyhmäksi on rajattu 7-15- vuotiaat peruskouluikäiset, sillä kyseinen ikä nuoren kognitiivisen kehityksen kannalta erityisen merkittävä ajanjakso.

\section{Tutkimuskysymys}

Miten videopelien pelaaminen vaikuttaa lasten kognitiivisiin taitoihin? 

\chapter{Tutkimusmenetelmä}
Tutkimusmenetelmänä tässä työssä käytetään kirjallisuukatsausta. Menetelmä on työhön sopiva, sillä aiheesta on tehty paljon aiempia tutkimuksia ja tämän menetelmän avulla tuloksia voidaan yhdistää laajemman ymmärryksen saavuttamiseksi. Kirjallisuuskatsaus mahdollistaa tutkimuksien keskenäisen vertailun, eikä näkökulma ole rajattu esimerkiksi kognitiivisten vaikutusten osalta vain tiettyyn osa-alueeseen. Metodi mahdollistaa myös mahdollisten lisätutkimuksien tarpeen kartoituksen, sillä tutkimusaiheen tutkimusaukkoja voidaan pohtia laaja-alaisesti.  

Tiedonhaku on toteutettu pääasiassa Google Scholar hakukoneella. Avainsanat hauissa ovat olleet: Children, Cognition, Videogames, Adolescent ja Learning. Tuloksista on rajattu pois tutkimukset, joissa tutkittavien ikä ei ole vastannut peruskouluikää. Tutkimusten lisäksi teoriapohjan tueksi on etsitty tilastotietoja ja kognitiivisen ymmärryksen tukena on käytetty pääosin kirjaa Mieli ja Aivot, Kognitiivisen neurotieteen oppikirja. (Hämäläinen, ym., 2006) 

\chapter{Merkitys}

Yksittäisistä tutkimuksista poiketen tämän katsauksen tarkoitus on ymmärtää videopelaamisen vaikutuksia laajempana kokonaisuutena. Tarkoituksena on luoda lukijalle laaja-alainen ymmärrys videopelaamisen vaikutuksista nuorten kognitiivisiin taitoihin. Jokaisella kognition osa-alueella on suuri merkitys lapsen oppimisen taitoihin joten laaja-alainen osaaminen rakentuukin näiden yksittäisten alueiden tarkastelun kautta.  

Videopelaamisen vaikutuksien ymmärtäminen on jokaiselle lapsen elämässä mukana olevalle kasvattajalle tärkeä taito. Vaikutusten ymmärtämisen kautta voidaan videopelaamista hyödyntää tehokkaammin positiivisena vaikutteena osana lapsen kasvua ja vähentää mahdollisia haittavaikutuksia. 

\printbibliography

\end{document}
