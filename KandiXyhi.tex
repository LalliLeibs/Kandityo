\documentclass[utf8,bachelor]{gradu3}
% Jos työ on kandidaatintutkielma eikä pro gradu, käytä ylläolevan asemesta
%\documentclass[utf8,bachelor]{gradu3}
% Jos kirjoitat englanniksi, käytä ylläolevan asemesta
%\documentclass[utf8,english]{gradu3}
% tai
%\documentclass[utf8,bachelor,english]{gradu3}

\usepackage{graphicx} % kuvien mukaan ottamista varten

\usepackage{amsmath} % hyödyllinen jos tekstisi sisältää matikkaa,
                     % ei pakollinen

\usepackage{booktabs} % hyvä kauniiden taulukoiden tekemiseen

% HUOM! Tämän tulee olla viimeinen \usepackage koko dokumentissa!
\usepackage[bookmarksopen,bookmarksnumbered,linktocpage]{hyperref}

\addbibresource{kandi.bib} % Lähdetietokannan tiedostonimi

\begin{document}

\title{Videopelien vaikutus peruskoululaisten kognitiivisiin taitoihin}
\author{Lauri Ahonen}
\contactinformation{\texttt{lauri.t.ahonen@student.jyu.fi}}
% jos useita tekijöitä, anna useampi \author-komento
%\supervisor{Ohjaamaton työ}
% jos useita ohjaajia, anna useampi \supervisor-komento

\subject{Kandidaatintutkielman}
%\type{tutkimussuunnitelma}

\maketitle

\mainmatter

\chapter{Johdanto}
Videopelaaminen on nykyajan nuorten keskuudessa suosittu vapaa-ajan harrastus. Vuoden 2022 pelaajabarometrin mukaan 10-19-vuotiaista nuorista 76\% pelaa digitaalisia pelejä viikoittain. \parencite{kinnunen2022pelaajabarometri} Kyseessä on siis laaja ilmiö, joka koskettaa suurinta osaa nuorista päivittäin. Laajan yleisön lisäksi videopelit ovat 2000-luvun aikana saavuttaneet ison roolin osana populaarikulttuuria ja ovatkin monen nuoren elämässä iso osa heidän identiteettiään. Teknologian kehityksen johdosta videopelaaminen on mahdollista lähes jokaiselle nuorelle, ja älypuhelimen omistajana pelit kulkevat aina käyttäjänsä mukana.  

Peruskouluaika on nuoren elämässä pitkä ja merkityksellinen elämänvaihe. ''Perusopetuksen tavoitteena on tukea oppilaiden kasvua ihmisinä ja yhteiskunnan jäseninä sekä opettaa tarpeellisia tietoja ja taitoja''.\parencite{OPH2025} Vaikka nykyaikana koulutusjärjestelmä onkin pyrkinyt muuttumaan akateemisesta oppimisesta laajemman ja moniulotteisemman osaamisen opetukseen, on perinteisten taitojen kuten lukemisen, laskemisen ja oppimisen taitojen opiskelu yhä koulutuksen keskiössä. Tämän oppimisen keskiössä on kognitiiviset taidot, jotka mahdollistavat tiedon syvällisen omaksumisen ja kriittisen ajattelun kehittymisen. 

Peruskouluikäiset lapset ovat kognitiivisen kehityksen kannalta kriittisessä elämänvaiheessa. Kognitio on moniulotteinen kokonaisuus, mutta tässä työssä käytän kognitiiviselle toiminnalle seuraavaa määritelmää: “Laajasti ymmärrettynä tiedonkäsittelyyn liittyvät toiminnot kuten havaitseminen, oppiminen, tarkkaivaisuus, muisti, päättely, sosiaalinen kognitio, kieli ja tietoisuus”. \parencite{hamalainen2006mieli} Nämä prosessit eivät kehity tyhjiössä, vaan vuorovaikutuksessa sisäisten sekä ulkoisten vaikutteiden johdosta. 

Tämän työn tarkoituksena on perehtyä siihen, kuinka videopelaaminen vaikuttaa peruskouluikäisten nuorten kognitiivisiin taitoihin. Tarkoituksena on kirjallisuuskatsauksen keinoin perehtyä aiheesta tehtyihin tutkimuksiin sekä tutkimuskirjallisuuteen ja muodostaa yleismaailmallinen käsitys aineistojen löydöksistä. Tässä työssä keskitytään oppimisen kannalta tärkeisiin kognition osa-alueisiin. Tutkimuskysymykset ovat seuraavat: 
\newpage
\begin{enumerate}
\item Kuinka säännöllinen videopelaaminen vaikuttaa peruskouluikäisten kognitiivisiin taitoihin? 
\item Miten videopelaamisen määrä tai pelin sisältö vaikuttaa kognitiivisiin taitoihin? 
\end{enumerate}

Lähdemateriaalina toimivat pääasiassa tutkimukset videopelien vaikutuksista kognition eri osa-alueisiin. Materiaalista on rajattu pois tutkimukset, joiden osallistujien ikä ei vastaa suomalaisen peruskoulujärjestelmän ikävuosia 7–16. Tämän lisäksi tutkimuksen päätarkoituksena on pitänyt olla videopelaamisen vaikutusten tarkastelu tiettyyn tai useampaan kognition osa-alueeseen. Kognition teoriakehyksen osalta lähteenä toimivat kehityspsykologiaa käsittelevät tieteelliset artikkelit sekä painetut kirjalliset lähteet.  Keskeisimpänä lähteiden etsinnän tietokantana on toiminut Google Scholar hakukone.  

\chapter{Videopelit ja peruskoululaiset}

\section{Videopelien historiasta}

\section{Mitä, miksi ja miten nuoret pelaavat?}

\chapter{Kognitiiviset taidot}
- Mitä ne ovat
- Miten kehittyvät
-Miksi erityisen tärkeitä koululaisille

\chapter{Videopelaamisen vaikutukset kognitiivisiin taitoihin}

Kuten tässä työssä on aikaisemmin todettu, videopelit ovat peruskouluikäisten keskuudessa suosittu ajanviete ja iso osa nuorisokulttuuria. Nuoren pääasiallinen rooli yhteiskunnassa on kuitenkin koulunkäynti. Koulutuksen kannalta onkin tärkeää tutkia nuorten elämän muita osa-alueita ja niiden vaikutuksia koulunkäyntiin. Tutkijat \textcite{kovess2016time} huomasivat artikkelissa “Is time spent playing video games associated with mental health, cognitive and social skills in young children?“, että yli 5 tuntia viikossa pelaavat nuoret pärjäsivät keskimääräistä paremmin koulussa ja osoittivat korkeampaa älyllistä toimintakykyä. Toisaalta artikkeli “Video gaming in school children: How much is enough?“ \parencite{pujol2016video} huomasi pelaamisen aiheuttavan keskittymishaasteita. Tutkimukset tuovatkin esiin tärkeän laajemman aiheen jota tässä luvussa on tarkoituksena käsitellä. Miten ja mihin peruskoululaisten kognitiivisiin taitoihin videopelit vaikuttavat? 

\section{Työmuisti}

Videopelaamisen positiivisia vaikutuksia työmuistiin on havaittu useissa tutkimuksissa. Tutkimuksessa “Association of Video Gaming With Cognitive Performance Among Children“ \parencite{jamanetworkopen} käytettiin fMRI-kuvantamista ja kognitiivisia testejä videopelaajien ja pelaamattomien 9-10-vuotiaiden lasten aivotoiminnan tutkimiseen. Testeissä huomattiin, että videopelaajat suoriuituivat 2-Back työmuistitehtävässä merkittävästi paremmin. Tämän lisäksi aivokuvantaminen havaitsi aktiivisempaa toimintaa työmuistiin liittyvillä aivoalueilla. Havaintoa tukee vuoden 2020 tutkimus, jossa videopelaajat suoriuituivat työmuistia testaavassa Digit Span -testissä merkittävästi paremmin \parencite{choudhury2022cognitive}. Molemmissa tutkimuksissa erot olivat tilastollisesti merkittäviä. Näihin havaintoihin perustuen voidaankin todeta, että videopelit voivat kehittää työmuistin toimintaa. 

Mitkä videopelien mekanismit saattavat mahdollistaa edellä mainitun vaikutuksen? Videopelit vaativat nopeaa tiedonkäsittelyä ja useiden tehtävien samanaikaista hallintaa. Tämä tarjoaa pelaajalleen kognitiivisia haasteita jotka vuorostaan harjaannuttavat käyttäjänsä työmuistia \parencite{choudhury2022cognitive}. Tämän lisäksi videopelit vaativat myös jatkuvaa tarkkaavaisuuden vaihtamista ja visuaalisten ärsykkeiden käsittelyä, joka voi harjaannuttaa pelaajan työmuistin kapasiteettia \parencite{jamanetworkopen}.  

Työmuistin kannalta pelin genrellä on vaikutusta sen tuomiin hyötyihin. Videopelaaminen yleisesti parantaa työmuistin toimivuutta, mutta pulma- ja strategiapelit kehittävät tätä kognition osa-aluetta tehokkaammin kuin muut peligenret.  Tämän lisäksi pelaajan yleinen taitotaso videopeleissä osoittaa tilastollista korrelaatiota työmuistin kehityksen kanssa. Mitä taitavampi pelaaja on, sitä paremmin hän suoruitui tutkimuksen muistitehtävissä. \parencite {zioga2024validation} 

Useat tutkimukset osoittavat että videopelit vaikuttavat postitiivisesti erityisesti lyhytkestoiseen työmuistiin. Löydöksiin on kuitenkin suhtauduttava varauksellisesti. Erityisesti työmuistin kehityksen siirtymisestä pitkäaikaisesti varsinaisiin oppimistilanteisiin ei ole tutkimusta. Tämän lisäksi peligenrellä ja pelaamiseen käytettävällä ajalla on suuri vaikutus saavutettaviin hyötyihin.   

\section{Visuaalinen hahmotus - ja käsittelykyky}

\section{Tarkkaavaisuus ja keskittymiskyky}


\chapter{Yhteenveto}

Haluaisin palautetta erityisesti Työmuisti kappaleesta. - Onko tarpeeksi selkeä -Onko lähteitä
käytetty järkevästi ja tukevatko ne toisiaan -Onko liikaa toistoa
Johdannosta voisi kertoa tuleeko motivaatio tämän tutkimuksen tekemiseksi tarpeeksi esiin
ja puuttuuko siitä jotain?
Lisäksi haluaisin kuulla miltä kognitiivisten taitojen jaottelu tuntuu lukijalle. Onko looginen tapa edetä vai pitäisikö asoita käsitellä ennemmin lähde kerrallaan?

\printbibliography

\end{document}
